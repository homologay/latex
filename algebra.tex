



\documentclass{article}

\usepackage[margin=1.5in]{geometry}
\usepackage{amsmath,amsthm,amssymb,hyperref}
\usepackage{physics}
\usepackage{upgreek}
\usepackage{xcolor}
\usepackage{halloweenmath}
\usepackage[dvipsnames]{xcolor}
\usepackage{tikz}
\usepackage{tikz-cd}

\newcommand{\R}{\mathbb{R}}  
\newcommand{\Z}{\mathbb{Z}}
\newcommand{\N}{\mathbb{N}}
\newcommand{\Q}{\mathbb{Q}}
\newcommand{\C}{\mathbb{C}}
\newcommand{\Log}{\text{Log}\:}
\newcommand{\Re}{\text{Re}\:}
\newcommand{\Im}{\text{Im}\:}
\newcommand{\im}{\text{im}\:}
\newcommand{\coker}{\text{coker}\:}
\newcommand{\Hom}{\text{Hom}}
\newcommand{\End}{\text{End}}

\newenvironment{theorem}[2][Theorem]{\begin{trivlist}
\item[\hskip \labelsep {\bfseries #1}\hskip \labelsep {\bfseries #2.}]}{\end{trivlist}}
\newenvironment{lemma}[2][Lemma]{\begin{trivlist}
\item[\hskip \labelsep {\bfseries #1}\hskip \labelsep {\bfseries #2.}]}{\end{trivlist}}
\newenvironment{claim}[2][Claim]{\begin{trivlist}
\item[\hskip \labelsep {\bfseries #1}\hskip \labelsep {\bfseries #2.}]}{\end{trivlist}}
\newenvironment{problem}[2][Problem]{\begin{trivlist}
\item[\hskip \labelsep {\bfseries #1}\hskip \labelsep {\bfseries #2.}]}{\end{trivlist}}
\newenvironment{proposition}[2][Proposition]{\begin{trivlist}
\item[\hskip \labelsep {\bfseries #1}\hskip \labelsep {\bfseries #2.}]}{\end{trivlist}}
\newenvironment{corollary}[2][Corollary]{\begin{trivlist}
\item[\hskip \labelsep {\bfseries #1}\hskip \labelsep {\bfseries #2.}]}{\end{trivlist}}

\newenvironment{solution}{\begin{proof}[Solution]}{\end{proof}}

\begin{document}

\large 

\vspace{0.05in}

1. Suppose $x \in B$ is integral over $A$, so $x^m + a_1x^{m-1} + ... + a_{m-1}x + a_m = 0$ for some $m\geq 1$ and $a_1,...,a_m \in A$. Thus we can write $x^n$ as an $A$-linear combination of $1, x, ..., x^{m-2}, x^{m-1}$, so $A[x] = \langle 1, x, ..., x^{m-2}, x^{m-1} \rangle$.
\\ \\ 
2. Suppose that $B = \langle b_1,...,b_n \rangle $ as an $A$-module. We first use the determinant trick to prove (special case of) the Cayley-Hamilton theorem, following the lecture notes and Theorem 4.3 in Eisenbud's book. The statement of the theorem with our notation is: \\ \\
If $\varphi \in \End_A(B)$ then there is a monic polynomial 
\[
p(x) = x^n + p_1 x^{n-1} + ... + p_{n-1}x+p_n
\]
such that $p \circ \varphi = 0$. \\ \\
For the proof, we let $\varphi\in \End_A(B)$ and note (as shown in the notes) that we can write $\varphi(b_i) = \sum a_{ij} b_j$, a sum of the generators with coefficients $a_{ij} \in A$. We consider $B$ as an $A[x]$-module by $\mu_x=\varphi$ (multiplication by $x$ is $\varphi$). Denoting the $n\times n$ identity matrix by $I$, the above means that 
\[
(xI - (a_{ij})_{ij})\begin{pmatrix} b_1\\
\vdots \\
b_n
\end{pmatrix}=0.
\]
Multiplying on the left by the cofactor matrix of $(xI - (a_{ij})_{ij})$ we get 
\[
(\underbrace{\det(xI-(a_{ij})_{ij})}_{\Delta})I\begin{pmatrix}b_1 \\
\vdots \\
b_n 
\end{pmatrix}=0.
\]
In other words, $\Delta b_i = 0$ for all $i =1,...,n$, so with $\Delta$ as our choice of $p(x)$, we see that $p\circ \varphi = 0$. \\ \\
From this, if $b \in B$, then choosing $\varphi = \mu_b$, multiplication by $b$, we have $p \circ \mu_b = 0$, ie. $b$ is integral over $A$.
\\ \\
3. Suppose $x, y \in B$ are integral over $A$, so $A[x]$ is finitely generated over $A$, and since $y$ is integral over $A$ it is integral over $A[x]$, so $A[x,y]$ is finitely generated over $A[x]$. As $A[x]$ is finitely generated over $A$, $A[x,y]$ is finitely generated over $A$. Thus by problem 2 every element of $A[x,y]$ is integral over $A$, so in particular $xy$ and $x-y$ are. As this is true for any two elements of $B$ integral over $A$, the integral closure of $A$ in $B$ is a subring of $B$.
\\ \\
4. (a) Suppose $x \in B$ is integral over $A$, so 
\[
x^m + a_1x^{m-1} + ... + a_{m-1}x + a_m = 0
\]
for some $m \geq 1$ and $a_i \in \mathfrak{a}$, where $i=1,...,m$. So, moving the degree less than $m$ terms over we see that
\[
x^m = \sum_{k=1}^{m-1} (-a_k)x^k \in \mathfrak{a}B.
\]
Thus $x \in \sqrt{\mathfrak{a}B}$. 
\\ \\
(b) We view $\sqrt{\mathfrak{a}B}$ as an $A$-module, and we denote $B = \langle b_1, ..., b_n \rangle$ over $A$, so in particular $\sqrt{\mathfrak{a}B}$ is an $A$-submodule of $\langle b_1,...,b_n\rangle$. We then follow the proof of problem 2, except we take the coefficients $a_{ij}$ to be in $\mathfrak{a}$ instead of $A$. This shows that each element of $\mathfrak{a}B$ is integral over $\mathfrak{a}$, and so if $x \in \sqrt{\mathfrak{a}B}$ then $x^n$ is integral over $\mathfrak{a}$, thus so is $x$.
\\ \\
5. We drop the function composition notation to save space, so $fg$ means $f \circ g$. We will show that $d^{n+1} d^n = 0$. Let $f = (f^j)_{j\in \Z} \in \Hom^n_A(C^{\bullet}, D^{\bullet})$, then
\begin{align*}
    d^{n+1}(d^n (f)) &= d^{n+1}( \underbrace{d_{D^{\bullet}}^{n+j} f^j +(-1)^{n+1} f^{j+1} d_{C^{\bullet}}^j }_{g^j})_{j\in \Z} \\
    &= (d_{D^{\bullet}}^{n+1+j}  g^j + (-1)^{n+2} g^{j+1}  d_{C^{\bullet}}^j)_{j\in \Z}.
\end{align*}
It suffices to show that the term above is zero for each $j$. This term is
\begin{align*}
    d_{D^{\bullet}}^{n+1+j} (d_{D^{\bullet}}^{n+j}  f^j + (-1)^{n+1} f^{j+1}  d_{C^{\bullet}}^j) + (-1)^{n+2}(d_{D^{\bullet}}^{n+1+j}  f^{j+1} +(-1)^{n+2} f^{j+2}  d_{C^{\bullet}}^{j+1})(d_{C^{\bullet}}^j)
\end{align*}
which is equal to
\[
d_{D^{\bullet}}^{n+1+j} d_{D^{\bullet}}^{n+j} f^j + (-1)^{n+1} d_{D^{\bullet}}^{n+1+j} f^{j+1} d_{C^{\bullet}}^j + (-1)^{n+2} d_{D^{\bullet}}^{n+1+j}f^{j+1}d_{C^{\bullet}}^j +(-1)^{2n+4} f^{j+2}d_{C^{\bullet}}^{j+1} d_{C^{\bullet}}^j
\]
and since the composition of differentials is zero, rearranging the signs we see the above is 
\[
(-1)^{n+1}(d_{D^{\bullet}}^{n+1+j}f^{j+1}d_{C^{\bullet}}^j - d_{D^{\bullet}}^{n+1+j}f^{j+1}d_{C^{\bullet}}^j) = 0.
\]
\\ \\
6. Suppose that $f = (f^j)_{j\in \Z} \in Z^0(\Hom^{\bullet}(C^{\bullet}, D^{\bullet}))$, so 
\[
0 = d^0(f) = (d_{D^{\bullet}}^j \circ f^j - f^{j+1} \circ d_{C^{\bullet}}^j)_{j\in \Z},
\]
which is equivalent to saying the following diagram commutes for each $j$, ie. $f$ is a cochain map. 
\[
\begin{tikzcd}
C^j \ar[d, "f^j"] \ar[r, "d_{C^{\bullet}}^j"] &C^{j+1} \ar[d, "f^{j+1}"] \\ D^j \ar[r, "d_{D^{\bullet}}^j"] &D^{j+1}
\end{tikzcd}
\]
So, $Z^0(\Hom^{\bullet}(C^{\bullet}, D^{\bullet})) \subset \Hom(C^{\bullet}, D^{\bullet})$, and since all the statements above work in the converse direction we have $Z^0(\Hom^{\bullet}(C^{\bullet}, D^{\bullet})) = \Hom(C^{\bullet}, D^{\bullet})$.
\\ \\
7. Let $f = (f^j)_{j\in \Z} \in B^0(\Hom^{\bullet}(C^{\bullet}, D^{\bullet})$. So, $f$ is a cochain map (problem 6) and $f = d^{-1}(g)$ for some $g \in \Hom^{-1}(C^{\bullet}, D^{\bullet})$. Note the $-1$ here denotes the index, not some inverse. 
As 
\[
d^{-1}(g) = (d_{D^{\bullet}}^{j-1} \circ g^j + g^{j+1}\circ d_{C^{\bullet}})_{j\in \Z}
\]
this is the same as saying as $f \sim 0$. So, 
\[
B^0(\Hom^{\bullet}(C^{\bullet}, D^{\bullet}) = \{ f \in \Hom(C^{\bullet}, D^{\bullet}) \: | \: f \sim 0 \}.
\]
\\ \\
8. This proof is adapted from the proof of proposition A3.12 in Eisenbud's book. Let $f,g \in \Hom(C^{\bullet}, D^{\bullet})$ with $f \sim g$, so $f-g \sim 0$. We first show that $H^n(f-g) = H^n(f) - H^n(g)$, so it suffices to show $H^n(f-g) = 0$ (for all $n$). This is immediate from the definition: let $x\in C^n$, then 
\begin{align*}
H^n(f)(x+B^n(C^{\bullet})) - H^n(g)(x+B^n) &= f(x) - g(x) + B^n(D^{\bullet}) \\ &= H^n(f-g)(x+B^n(C^{\bullet})).
\end{align*}
So, as $f-g \sim 0$ there exists a $k = (k^n)_n \in \Hom^{-1}(C^{\bullet}, D^{\bullet})$ such that $d_{D^{\bullet}}^{n-1} \circ k^n + k^{n+1}\circ d_{C^{\bullet}}^n = (f-g)^n$ for all $n$. Thus if $z \in Z^n(C^{\bullet})$ we have 
\begin{align*}
    (f-g)^n(z) &= (d^{n-1}_{D^{\bullet}} \circ k^n)(z) + (k^{n+1} \circ d_{C^{\bullet}}^n)(z) \\
    &= (d_{D^{\bullet}}^{n-1}\circ k^n)(z) + k^{n+1}(0) \\
    &= (d^{n-1}_{D^{\bullet}} \circ k^n)(z) \in B^n(D^{\bullet})
\end{align*}
so $H^n(f-g)$ is the zero map on homology for all $n$.

\end{document}
